\documentclass{article}

% Language setting
% Replace `english' with e.g. `spanish' to change the document language
\usepackage[english]{babel}
\setcounter{section}{-1}
% Set page size and margins
% Replace `letterpaper' with`a4paper' for UK/EU standard size
\usepackage[letterpaper,top=2cm,bottom=2cm,left=3cm,right=3cm,marginparwidth=1.75cm]{geometry}

\usepackage{mathtools}  
\usepackage{diffcoeff}  

% Useful packages
\usepackage{amsmath}
\usepackage{graphicx}
\usepackage[colorlinks=true, allcolors=blue]{hyperref}
\usepackage{makeidx}
\date{2021/2022}
\title{Progetto di Controlli Automatici T, gruppo A \\
Progetto Tipologia C - Traccia 1 \\
---\\
Controllo del motore di un'automobile 
}
\author{Andrea Belano\\ Gabriele Ceccolini\\ Filippo Loddo\\ Simone Merenda}

\begin{document}
\maketitle
\begin{figure}[htp]
    \centering
    \includegraphics[width=12cm]{dio.png}
\end{figure}

\newpage
\renewcommand*\contentsname{Indice}
\tableofcontents
\newpage
\section{Introduzione e specifiche}

Si consideri la modellazione di un motore a scoppio con una massa d'aria nel collettore di aspirazione del pistone pari a $m(t)$ e una velocità angolare dell'albero di trasmissione pari a $\omega(t)$. 
\subsection{Dinamica del sistema}
Consideriamo inoltre la dinamica del sistema composta dalle seguenti equazioni differenziali: 
\begin{subequations}
\begin{equation}\label{1a}
    \dot{m} = \gamma_1(1-\cos(\beta\theta - \psi))-\gamma_2\omega m 
\end{equation}
\begin{equation}
    J\dot{\omega} = \delta_1 m - \delta_2 \omega - \delta_3 \omega^3
\end{equation}
\end{subequations}
Dove: \newline
\begin{itemize}
    \item $\theta(t)$ rappresenta l'angolo di accelerazione
    \item $\gamma_1(1-\cos(\beta\theta - \psi))$ modella la caratteristica intrinseca della valvola.
    \item $J$ rappresenta il momento d'inerzia equivalente del sistema automobile.
    \item $\delta_1 m$ descrive la coppia trasmessa all'albero motore.
    \item $\delta_2 \omega$, con  modella l'attrito nel motore.
    \item $\delta_3 \omega^2$ descrive la
resistenza dell'aria.
\end{itemize} 
Con $\gamma_1,\psi,J, \delta_1,\delta_2,\delta_3 \in  \mathbb{R}$. \newline
\linebreak
Si suppone inoltre di potere misurare la velocità angolare dell'albero di trasmissione $\omega(t)$.

\section{Linearizzazione del sistema}
\subsection{Dinamica in forma di stato}
Per prima cosa si vuole portare il sistema (1) nella forma di stato
\begin{subequations}
\begin{equation}
    \dot{x} = f(x,u)
\end{equation}
\begin{equation}
    y=h(x,u).
\end{equation}
\end{subequations}
Definiamo la variabile di stato come 
\begin{equation}
    x=\begin{pmatrix}
x_1\\
x_2
\end{pmatrix} = \begin{pmatrix}
m\\
\omega
\end{pmatrix}
\end{equation}
Imponiamo $\theta$ e $\omega$ rispettivamente come variabile d'ingresso e di uscita
\begin{subequations}
\begin{equation}
    \theta = u
\end{equation}
\begin{equation}
    \omega = y
\end{equation}
\end{subequations}

La funzione di stato di $f(x)$ si presenta nella forma
\begin{subequations}
\begin{equation}
    \dot{x} = \begin{pmatrix}
\dot{x_1}\\
\dot{x_2}
\end{pmatrix} = 
\begin{pmatrix}
\gamma_1 - \gamma_1 cos(\beta u(t) - \psi) - \gamma_2 x_1 x_2 \\
\frac{\delta_1}{J}x_1 - \frac{\delta_2}{J}x_2 - \delta_3 x_2^2 
\end{pmatrix} = \\
\begin{pmatrix}
f_1(x,u) \\
f_2(x,u)
\end{pmatrix} = f(x,u)
\end{equation}
\begin{equation}
    y(t)=h(x,u)=x_2(t)
\end{equation}
\end{subequations}
Notare come la funzione di uscita $y(t)$ dipende solo da $x_2$ in quanto $y(t)=\omega(t)$.
\subsection{Ricerca coppia di equilibrio}
A questo punto partendo dal valore di equilibrio della pulsazione $\omega_e$ si deve trovare l'intera coppia di equilibrio $(x_e, u_e)$.
\newline Per fare questo riscriviamo le equazioni (5) sostituendo $x_2$ con il corrispettivo equilibrio $\omega_e$
\begin{equation}
    x_{2e} =\omega_e = 10
\end{equation}\linebreak Similmente le costanti verranno sostituite con i valori indicati nella tabella
\begin{center}
   \begin{tabular}{ |c|c| } 
 \hline
 \gamma_1 & 0,5 \\ 
 \hline
 \gamma_2 & 0,1 \\ 
 \hline
 \beta & 1,1 \\ 
 \hline
 \psi & 0,02 \\ 
 \hline
 \delta_1 & 5\cdot10^4 \\ 
 \hline
 \delta_2 & 0,1 \\ 
 \hline
 \delta_3 & 0,01 \\
 \hline
 J  & 40 \\
 \hline
 \omega_e & 10 \\
 \hline
\end{tabular}
\end{center}
Otteniamo quindi un sistema di due equazioni in due incognite ($x_{1e} , u_e$)
\begin{equation}
    \begin{cases}
    0,5 - 0,5 \cos{1,1 u_e - 0,02} - 0,1  x_{1e} 10 = 0 \\
    \frac{50 \cdot 10^4}{40} x_{1e}- \frac{0,1}{40}10 - \frac{0,01}{40}10^2 = 0
    \end{cases}
\end{equation}
Risolvendo il sistema otteniamo i valori di equilibrio cercati
\begin{equation}
    \begin{cases}
    x_{1e} =0,00004\\
    u_e = 0,03 
    \end{cases}
\end{equation}
In particolare l'equazione da cui si ricava la $u_e$
\begin{equation}
    0,5 -0,5\cos{1,1u_e-0,02}-\frac{1}{25000} = 0
\end{equation}
ha due zeri (oltre alle infinite soluzioni periodiche), in particolare uno negativo ($u_e = -0,007,u_e=0,03$).
Scegliamo arbitrariamente lo zero positivo $u_e=0,03$.
Infine approssimando $x_{1e}$ a 0, otteniamo la seguente coppia di equilibri.
\begin{equation}
    (x_e, u_e) = ( (x_{1e} = 0, x_{2e}= 10) , u_e = 0,03)) 
\end{equation}
\subsection{Linearizzazione del sistema non lineare nell'equilibrio }
Avendo trovato la coppia di equilibrio, procedo nel linearizzare il sistema non lineare (2), cosi da attenere un sistema linearizzato del tipo
\begin{subequations}
\begin{equation}
    \delta\dot{x}=A\delta x + B\delta u
\end{equation}
\begin{equation}
    \delta y=C\delta x + D\delta u
\end{equation}
\end{subequations}
Per trovare $\delta \dot{x_e}$ imposto l'equazione alle derivate parziali
\begin{equation}
    \delta \dot{x_e} = \diffp{f}{x}_{\substack{x=(0,10)\\ u=0,03}} \delta x(t) + \diffp{f}{u}_{\substack{x=(0,10)\\ u=0,03}} \delta u(t)
\end{equation}
Calcolo la Jacobiana $\diffp{f}{x}$
\begin{equation}
   \diffp{f}{x}=\begin{pmatrix}
   \diffp{f_1}{x_1} & \diffp{f_1}{x_2} \\
   \diffp{f_2}{x_1} & \diffp{f_2}{x_2}
   \end{pmatrix} = \begin{pmatrix}
   -0,1 x_2 & -0,1 x_1 \\
   1250 & -0,1/40-0,02 x_2
   \end{pmatrix}
\end{equation}
\newpage
Ora calcolando la Jacobiana nell'equilibrio ottengo la matrice $A_e$
\begin{equation}
    A_e = \diffp{f}{x}_{\substack{x=(0,10)\\ u=0,03}}=\begin{pmatrix}
    -1 & 0 \\
    1250 & -0,2
    \end{pmatrix}
\end{equation}
Ora calcolando gli autovalori di $A_e$
\begin{subequations}
\begin{equation}
    \det\begin{pmatrix}
    -1-\lambda & 0 \\
    1250 & -0,2-\lambda
    \end{pmatrix}=0
\end{equation}
\begin{equation}
    \lambda^2 + 1,2\lambda + \frac{1}{5}=0
\end{equation}
\begin{equation}
    \lambda_{1,2} = (-1, -0,2)
\end{equation}
\end{subequations}
Ottengo (-1, -0,2) come autovalori. Possiamo dedurre quindi che il sistema è \emph{asintoticamente stabile} poichè ha tutti gli autovalori con parte reale negativa.
\\
\linebreak
Similmente calcolo la Jacobiana $\diffp{f}{u}$
\begin{equation}
   \diffp{f}{u}=\begin{pmatrix}
   \diffp{f_1}{u}\\
   \diffp{f_2}{u}
   \end{pmatrix} = \begin{pmatrix}
   0,55 \sin(1,1 u -0,02) \\
   0
   \end{pmatrix}
\end{equation}
Calcolando la Jacobiana nell'equilibrio ottengo la matrice $B_e$
\begin{equation}
    B_e = \diffp{f}{u}_{\substack{x=(0,10)\\ u=0,03}}=\begin{pmatrix}
    0,007 \\
    0
    \end{pmatrix}
\end{equation}

Similmente calcolo $C_e$
\begin{equation}
    C_e = \diffp{h}{x}_{\substack{x=(0,10)\\ u=0,03}}=\begin{pmatrix}
    0 & 1
    \end{pmatrix}
\end{equation}

La matrice $D_e$ invece è nulla.
\begin{equation}
    D_e = 0
\end{equation}

\section{Funzione di trasferimento}
E' necessario calcolare la funzione di trasferimento da $\delta u$ a $\delta y$, ovvero la funzione $G(s)$ tale che
\begin{equation}
    \delta Y(s) = G(s)\delta U(s)
\end{equation}
Ricordiamo che la funzione di trasferimento $G(s)$ si può scrivere come
\begin{equation}
    G(s) = C(sI - A)^{-1} B + D
\end{equation}
Quindi sostituisco con le matrici calcolate in precedenza 
\begin{subequations}
\begin{equation}
    G(s)=\begin{pmatrix}
    0 & 1
    \end{pmatrix}
    \begin{pmatrix}
    s+1 & 0 \\
    -1250 & s+0,2
    \end{pmatrix}^{-1}
    \begin{pmatrix}
    0,007 \\
    0
    \end{pmatrix} =
\end{equation}
\begin{equation}
    = \frac{\begin{pmatrix}
    0 & 1
    \end{pmatrix} 
    adj\begin{pmatrix}
         s+1 & 0 \\
    -1250 & s+0,2
    \end{pmatrix}  \begin{pmatrix}
    0,007 \\
    0
    \end{pmatrix}}
    {\det\begin{pmatrix}
        s+1 & 0 \\
    -1250 & s+0,2
    \end{pmatrix}}
\end{equation}
Calcolo il determinante al denominatore
\begin{equation}
    \det\begin{pmatrix}
     s+1 & 0 \\
    -1250 & s+0,2
    \end{pmatrix} = (s+1)(s+0,2)
\end{equation}
Calcolo la matrice aggiunta
\begin{equation}
    adj\begin{pmatrix}
         s+1 & 0 \\
    -1250 & s+0,2
    \end{pmatrix} =
    \begin{pmatrix}
    s+0,2 & 1250 \\
    0 & s+1
    \end{pmatrix}^{T} =
    \begin{pmatrix}
         s+0,2 & 0 \\
         1250 & s+1
    \end{pmatrix}
\end{equation}

Possiamo quindi riscrivere $G(s)$ come
\begin{equation}
    G(s)= \frac{\begin{pmatrix}
    0 & 1
    \end{pmatrix} 
    \begin{pmatrix}
         s+0,2 & 0 \\
    1250 & s+1
    \end{pmatrix}  \begin{pmatrix}
    0,007 \\
    0
    \end{pmatrix}}
    {(s+1)(s+0,2)}=
\end{equation}
\begin{equation}
    = \frac{\begin{pmatrix}
    1250 & s+1
    \end{pmatrix} 
    \begin{pmatrix}
    0,007 \\
    0
    \end{pmatrix}}
    {(s+1)(s+0,2)} = 
\end{equation}
\begin{equation}
=\frac{8,75}{(s+1)(s+0,2)}=    
\end{equation}
Moltiplico numeratore e denominatore per 5
\begin{equation}
   = \frac{43,75}{(s+1)(5s+1)}
\end{equation}
\end{subequations}




\newpage
\section{Progettazione regolatore}
\subsection{Prestazioni Statiche}
\subsection{Prestazioni Dinamiche}

\section{Test del regolatore sul sistema non linearizzato}
\section{Punti opzionali}

\end{document}